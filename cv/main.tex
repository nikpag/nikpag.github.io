\documentclass[letterpaper, 12pt]{resume}

\usepackage{libertinus}
\usepackage[colorlinks = true,
            linkcolor = blue,
            urlcolor  = blue,
            citecolor = blue,
            anchorcolor = blue]{hyperref}

\name{Nikolaos Pagonas}

% You can use the \address command up to 3 times for 3 different addresses or pieces of contact information
% Any new lines (\\) you use in the \address commands will be converted to symbols, so each address will appear as a single line.

\address{
    \href{mailto:nikospagonas00@gmail.com}{nikospagonas00@gmail.com} \\
    \href{https://nikpag.github.io}{nikpag.github.io}
}

\begin{document}

% TODO Add Minion Pro, Myriad Pro, and Inconsolatas as fonts
% TODO Add \TODO macro

\begin{rSection}{Education}
    \begin{rSubsection}{Columbia University}{Sep 2024 --- Present}{PhD in Computer Science}{New York, NY, US}
        \item \textbf{Advisor:} Prof.\ Kostis Kaffes
    \end{rSubsection}

    \begin{rSubsection}{National Technical University of Athens}{Oct 2018 --- Jun 2024}{Master of Electrical and Computer Engineering}{Athens, Greece}
        \item \textbf{GPA:} 9.40/10.00 (top 2\% of cohort)
        % \item \textbf{Specialization:} Computer Science
        \item \textbf{Related courses:} Distributed Systems, Parallel Processing, Operating Systems, Computer Networks
        \item \textbf{Thesis:} \textsc{SPLaSh}: Scaling Out Shell Scripts on Serverless Platforms
        \item \textbf{Advisors:} Prof.\ Georgios Goumas (NTUA), Prof.\ Nikos Vasilakis (Brown University)
    \end{rSubsection}
\end{rSection}

\begin{rSection}{Research and Work Experience}
    \begin{rSubsection}{Brown University}{Sep 2023 --- May 2024}{Student Intern}{Providence, RI, US}
        \item \textbf{Advisor:} Prof.\ Nikos Vasilakis
        \item Conducted research on improving the performance of shell scripts by running them on serverless infrastructure
    \end{rSubsection}

    \begin{rSubsection}{National Technical University of Athens}{Oct 2022 --- Sep 2023}{Technical Support}{Athens, Greece}
        \item \textbf{Supervisors:} Prof.\ P. Tsanakas, Prof.\ E.\ D. Sykas
        \item Provided technical support for the university's OpenStack cloud infrastructure, as well as its coursework platform, both used extensively by all nine departments
    \end{rSubsection}

    \begin{rSubsection}{Arrikto Inc.}{Dec 2022 --- Mar 2023}{Software Engineering Intern}{Athens, Greece}
        \item Explored how the company can serve PyTorch models more efficiently by using KServe, a highly scalable machine learning deployment platform for Kubernetes
    \end{rSubsection}

    % \begin{rSubsection}{Seed Robotics}{Oct 2022 - Jun 2023}{Volunteer Robotics Instructor}{Athens, Greece}
    %     \item Guided school students through construction and programming of a plethora of robotics projects
    %     \item Served as coordinator and leader of the school robotics team in the annual municipal festival on science, arts, and sports
    % \end{rSubsection}
\end{rSection}

\begin{rSection}{Teaching Experience}
    \begin{rSubsection}{Brown University}{Spring 2024}{Teaching Assistant}{}
        \item \textbf{Course:} Distributed Systems
        \item \textbf{Professor:} N. Vasilakis
        \item Helped revamp the course by redesigning its semester-long project, and by implementing infrastructure for autograding distributed systems
    \end{rSubsection}

    \begin{rSubsection}{National Technical University of Athens}{Spring 2022}{Lab Assistant}{}
        \item \textbf{Course:} Operating Systems
        \item \textbf{Professors:} N. Koziris, G. Goumas
    \end{rSubsection}

    \begin{rSubsection}{National Technical University of Athens}{Spring 2020}{Lab Assistant}{}
        \item \textbf{Course:} Programming Techniques
        \item \textbf{Professors:} N. Papaspyrou, A. Pagourtzis, G. Stamou, G. Goumas
    \end{rSubsection}

    \begin{rSubsection}{National Technical University of Athens}{Fall 2019}{Lab Assistant}{}
        \item \textbf{Course:} Introduction to Programming
        \item \textbf{Professors:} S. Zachos, N. Papaspyrou, D. Fotakis
    \end{rSubsection}
\end{rSection}

% \begin{rSection}{Publications}
%     NONE
% \end{rSection}

% \begin{rSection}{Volunteering Experience}
%     NONE
% \end{rSection}

\begin{rSection}{Service}
    \begin{rSubsection}{17th European Workshop on Systems Security --- EuroSec 2024}{2024}{Web and Publication Chair}{}
        \item[]
    \end{rSubsection}
    \vspace{-4ex}
\end{rSection}

\begin{rSection}{Honors and Awards}
    \begin{rSubsection}{Columbia University Presidential Fellowship}{2024}{Fu Foundation School of Engineering and Applied Science}{}
        \item \textbf{Selection rate:} 10/700 (1.4\%)
    \end{rSubsection}
\end{rSection}

% \begin{rSection}{Competitions}
%     \begin{rSubsection}{14th National Electrical and Computer Engineering Conference}{Apr 2023}{Volos, Greece}{}
%         \item Won 1st place in the conference-wide programming competition
%     \end{rSubsection}

%     \begin{rSubsection}{IEEEXtreme}{2019-2022}{}{}
%         \item Annual international programming competition held by IEEE
%         \item \textbf{Proctor:} Prof. N. Koziris
%         \item Ranked 14th in Greece, 91st in Europe
%     \end{rSubsection}
% \end{rSection}

% \begin{rSection}{Seminars}
%     \begin{rSubsection}{Mega-ACE Blockchain Seminar}{2023}{}{}
%         \item Lectures and workshops on blockchain, cryptography and smart contracts
%         \item Co-organized by Purdue University (Prof. V. Zikas) and the National Technical University of Athens (Prof. A. Pagourtzis)
%     \end{rSubsection}
% \end{rSection}

% \begin{rSection}{Projects}
%     \begin{rSubsection}{\href{https://github.com/nikpag/myCharts}{myCharts - Charts as a Service}}{Spring 2023}{}{}
%         \item Web application that allows creating a variety of charts out of CSV data. Built with Docker and Apache Kafka in order to provide seamless deployment and high scalability.
%     \end{rSubsection}

%     \begin{rSubsection}{\href{https://github.com/nikpag/Mockchain}{Mockchain}}{Fall 2022}{}{}
%         \item Blockchain-based cryptocurrency system implemented in Go and deployed on the cloud.
%     \end{rSubsection}

%     \begin{rSubsection}{\href{https://github.com/nikpag/KBeans}{KBeans}}{Fall 2022}{}{}
%         \item Implementation of the k-means algorithm in OpenMP, CUDA, and MPI. Accompanied by thorough benchmarking in order to compare the three programming models.
%     \end{rSubsection}

%     \begin{rSubsection}{\href{https://github.com/nikpag/encryptIOn}{encryptIOn}}{Fall 2021}{}{}
%         \item Terminal-based encrypted chat built with TCP/IP sockets for communication, cryptodev-linux for encryption, and virtIO for virtualization support.
%     \end{rSubsection}

% \end{rSection}


\begin{rSection}{Skills}

    \begin{tabular}{@{} >{\bfseries}l @{\hspace{6ex}} l @{}}
        Programming Languages & C, Python, Javascript, Shell, Java, Go              \\
        Platforms             & AWS, Docker, Kubernetes, Apache Kafka, Apache Spark \\
        Tools                 & Git, LaTeX
    \end{tabular}

\end{rSection}

% \begin{rSection}{Honors and Awards}
% \end{rSection}

\begin{rSection}{Languages}
    \begin{tabular}{@{} >{\bfseries}l @{\hspace{6ex}} l @{}}
        English & Fluent --- TOEFL iBT 116/120 (C2) \\
        Greek   & Native                            \\
    \end{tabular}
\end{rSection}

% \begin{rSection}{References}
% \end{rSection}

% \begin{rSection}{Last Updated}
%     \today
% \end{rSection}

\end{document}

Talks:
SPLaSh: Towards serverless shell scripting
Brown, DSI building
3rd PaSh workshop, 20231018

SPLaSh: Scaling Out Shell Scripts Using Serverless
Brown, CIT building
Sysread, 20231110

SPLaSh: Scaling out shell scripts on serverless platforms
Brown, CIT building
Sysread (SOSP pitch), 20240207

SPLaSh: Scaling out shell scripts on serverless platforms
Brown, DSI building
PaSh retreat (Spring 2024), Apr 25 2024
